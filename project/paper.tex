\documentclass{article}
\usepackage{graphicx,fancyhdr,amsmath,amssymb,amsthm,subfig,url,hyperref,enumitem,etoolbox}
\usepackage{algorithm}% http://ctan.org/pkg/algorithms
\usepackage{algpseudocode}% http://ctan.org/pkg/algorithmicx
\usepackage[margin=0.9in]{geometry}

%----------------------- Macros and Definitions --------------------------

%%% FILL THIS OUT
\newcommand{\studentname}{Keyan Pishdadian}
\newcommand{\uwid}{keyanp}
%%% END



\renewcommand{\theenumi}{\bf \Alph{enumi}}

%\theoremstyle{plain}
%\newtheorem{theorem}{Theorem}
%\newtheorem{lemma}[theorem]{Lemma}

\fancypagestyle{plain}{}
\pagestyle{fancy}
\fancyhf{}
\fancyhead[RO,LE]{\sffamily\bfseries\large University of Washington}
\fancyhead[LO,RE]{\sffamily\bfseries\large  Incentives in Computer Science}
\fancyfoot[LO,RE]{\sffamily\bfseries\large \studentname: \uwid @uw.edu}
\fancyfoot[RO,LE]{\sffamily\bfseries\thepage}
\renewcommand{\headrulewidth}{1pt}
\renewcommand{\footrulewidth}{1pt}

\graphicspath{{figures/}}

%-------------------------------- Title ----------------------------------

\title{\textbf{Improving healthcare outcomes and cost through analysis and design of provider incentives}}
\author{\studentname \qquad Student ID: \uwid}

%--------------------------------- Text ----------------------------------

\begin{document}
\maketitle

\section*{Abstract}

\paragraph{} Healthcare is a socially and economically important aspect of the modern United States. Roughly 1/6th of US consumer spending and ~48\% of federal \cite{federal_spend} spending goes towards some form of healthcare \cite{econharvard} and the success, efficiency, and outcomes of this market reflect directly on the viability and happiness of American citizens and the US economy. The decision making of physicians plays a critical role in this system, with roughly 80\% of all expenditure being a result of physicians' decisions \cite{trust}. Despite this critical role, the incentive structures that underlie provider decision making are poorly designed both financially and from a provider risk perspective. The classic incentive structures used result in a principle-agent problem between providers (agents) and payers (government/insurance companies), as well as multiple ``prisoners' dilemmas" between providers and patients as well as providers and other providers \cite{blended}. In this paper we analyze the inefficiencies and sub-optimal equilibria that result from the use of classic incentive systems, then extend recent ideas to propose a hybrid incentive structure that increases provider profits and improves patient outcomes.

\section*{Introduction}

\paragraph{}

\section*{Outline}

\begin{itemize}
    \item Abstract
        \begin{itemize}
            \item See working abstract above
        \end{itemize}
    \item Introduction
        \begin{itemize}
            \item General background and overview of actors (patients, insurers, providers)
            \item Discussion of incentives for actors and present how actors affect each other through a `influence diagram'
            \item Formal framework for structuring incentive systems, adapted lightly from \cite{blended}
        \end{itemize}
    \item Background, discussion of current classic incentive systems
        \begin{itemize}
            \item "Fee-for-service" model, benefits, problems, equilibria
            \item "Capitation" model, benefits, problems, equilibria
            \item Review of alternative strategies used in the wild, principally Accountable Care Organizations (ACOs), their analysis and problems. Leaning heavily on analysis provided by \cite{msdt}.
        \end{itemize}
    \item Hybrid Approach Proposal
        \begin{itemize}
            \item Not fully fleshed out, but uses a combination of ideas from ACOs \cite{mdst}, incorporation of trust \cite{trust} and externalities \cite{blended} to create new payoff matrices for the major agent interactions
            \item Analysis using formal framework discussed in introduction
        \end{itemize}
\end{itemize}

%--------------------------------- Bibliography ----------------------------------

\begin{thebibliography}{9}

\bibitem{econharvard}
Mankiw NG. (2017) The Economics of Healthcare.

\bibitem{federal_spend}
The True Cost of Health Care: An Analysis of Americans’ Total Health Care Spend
\\\texttt{https://a2e0dcdc-3168-4345-9e39-788b0a5bb779.filesusr.com/ugd/29ca8c_4975d361740b434b88649860658f9e8f.pdf}

%6
\bibitem{trust}
Djulbegovic, Benjamin \& Hozo, Iztok \& Ioannidis, John. (2014). Modern health care as a game theory problem. European Journal of Clinical Investigation. 45. 10.1111/eci.12380.

% 1
\bibitem{inflation}
Agee, M.D., Gates, Z. (2013). Lessons from Game Theory about Healthcare System Price Inflation. Appl Health Econ Health Policy 11, 45–51. https://doi.org/10.1007/s40258-012-0003-z

%7
\bibitem{tarrant}
Tarrant, C., Stokes, T., \& Colman, A. M. (2004). Models of the medical consultation: opportunities and limitations of a game theory perspective. Quality \& safety in health care, 13(6), 461–466. doi:10.1136/qhc.13.6.461kj

%4
\bibitem{blended}
DeVoe, J. E., \& Stenger, R. (2013). Aligning provider incentives to improve primary healthcare delivery in the United States. OA family medicine, 1(1), 7. doi:10.13172/2052-8922-1-1-958

%5
\bibitem{msdt}
Zhang, H., Wernz, C. \& Slonim, A.D. (2016). Aligning incentives in health care: a multiscale decision theory approach. EURO J Decis Process 4, 219–244. https://doi.org/10.1007/s40070-015-0051-3

%2
\bibitem{mindyour}
Prisoners dilemma and doctor prescribing
\\\texttt{https://mindyourdecisions.com/blog/2009/08/18/how-to-improve-health-care-using-game-theor\\y-the-prisoners-dilemma/}

\end{thebibliography}

%-------------------------------------------------------------------

\end{document}
